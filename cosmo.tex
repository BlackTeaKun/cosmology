\documentclass[a4paper,UTF8,fleqn]{book}
\usepackage{geometry}
\geometry{left=2.0cm, right=2.0cm, top=2.5cm, bottom=2.5cm}

\usepackage{ctex}

\usepackage{physics}

\usepackage[colorlinks, linkcolor=black]{hyperref}
\usepackage{amsmath}
\numberwithin{equation}{section}


\usepackage{multicol}
\columnseprule=1pt

% \usepackage{pgfplots}

\setlength{\parindent}{2em}
\setlength{\parskip}{0.5\baselineskip}

\title{习题}
\date{}
\begin{document}
\allowdisplaybreaks[3]
\maketitle
\tableofcontents
\chapter{The Smooth, Expanding Universe}
\begin{itemize}
    \item 测地线方程
    \begin{equation}
        \dv[2]{x^l}{t} + \Gamma^l_{jk}\dv{x^j}{t}\dv{x^k}{t} = 0\label{geodesic}
    \end{equation}
    \begin{equation}
        \dv[2]{x^\mu}{\lambda} + \Gamma^\mu_{\alpha\beta}\dv{x^\alpha}{\lambda}\dv{x^\beta}{\lambda} = 0\label{4dgeodesic}
    \end{equation} 
    \item 克里斯托弗记号(克氏符) 
    \begin{equation}
        \Gamma^\mu_{\alpha\beta} = \frac{g^{\mu\nu}}{2}\qty[\pdv{g_{\alpha\nu}}{x^\beta} + \pdv{g_{\beta\nu}}{x^\alpha} - \pdv{g_{\alpha\beta}}{x^\nu}]\label{chris}
    \end{equation} 
    \item FRW度规
    \begin{equation}
        g_{\mu\nu} = \mqty(\dmat[0]{-1, a(t)^2, a(t)^2, a(t)^2}) \label{FRWmertic}
    \end{equation}
    \item Ricci 张量
    \begin{equation}
        R_{\mu\nu} = \Gamma^\alpha_{\mu\nu, \alpha} - \Gamma^\alpha_{\mu\alpha, \nu} + \Gamma^\alpha_{\beta\alpha}\Gamma^\beta_{\mu\nu} - \Gamma^\alpha_{\beta\nu}\Gamma^\beta_{\mu\alpha}\label{riccit}
    \end{equation} 
    \item Ricci 标量
    \begin{equation}
        \mathcal{R} = R_{\mu\nu}g^{\mu\nu}
    \end{equation}
    \item 爱因斯坦场方程
    \begin{equation}
        G_{\mu\nu}\equiv R_{\mu\nu} - \frac{1}{2}g_{\mu\nu}\mathcal{R} = 8\pi GT_{\mu\nu}\label{einsteineq}
    \end{equation}
\end{itemize}
\section{Exercise 1.}
\begin{align*}
    T_0 = 2.725K &\to eV\\ 
        &=8.617342 \times 10^{-5} \times 2.725\  eV = 2.348 \times 10 ^{-4} \ eV \\
    \rho_\gamma = \pi^2 T_0^4/15 &\to eV^4 \\
        & = \frac{\pi^2}{15} (2.348\times 10^{-4})^4\ eV = 2.00 \times 10 ^{-15}eV^4 \\
        & \to g\, cm^{-3}(\frac{ET^2}{L^5}) \\
        & =\frac{\pi^2(k_BT_0)^4}{15c^5\hbar^3} = \frac{2.00 \times 10 ^{-15}}{(1.9732 \times 10^{-5})^3} \frac{eV}{cm^{-3}}\frac{1}{c^2} = 4.633 \times 10^{-34}g\, cm^{-3}\\
    1/H_0 &\to cm \\
        & = \frac{1}{2.133h\times 10^{-42}}\frac{\hbar c}{GeV} = \frac{1.973\times 10 ^{-14}}{2.133h\times 10^{-42}}\, cm = 9.25h^{-1} \times 10^{27}\, cm\\
    m_{Pl} &\equiv 1.2\times 10 ^{19}\, GeV \to K\\
        &\to K\\
        & = \frac{1.2\times 10 ^{19}\, GeV}{k_B} = 1.39 \times 10 ^{32}K\\
        &\to cm^{-1}\\
        & = \frac{1.2\times 10 ^{19}\, GeV}{c\hbar} = 6.09\times 10 ^{32} cm^{-1}\\
        &\to sec^{-1}\\
        & = \frac{1.2\times 10 ^{19}\, GeV}{\hbar} = 1.823 \times 10^{43} sec^{-1}
\end{align*}
% \section{Exercise 2.}
% \paragraph{(a)}
$x^i = \{r, \theta\}, g_{ij} = \mqty(1 & 0 \\ 0 & r^2), g^{ij} = \mqty(1 & 0 \\ 0 & 1/r^2)$, 利用公式\ref{chris}, 先计算$\Gamma^1_{ij}:$
\begin{align*}
    \Gamma^1_{ij} &= \frac{g^{1 k}}{2}\qty[\pdv{g_{ik}}{x^j} + \pdv{g_{jk}}{x^i} - \pdv{g_{ij}}{x^k}]\\
    & = \frac{g^{1 1}}{2}\qty[\pdv{g_{i1}}{x^j} + \pdv{g_{j1}}{x^i} - \pdv{g_{ij}}{x^1}]
\end{align*}
当$i=j=1$时, $\Gamma^1_{11}=g^{11}\qty[\partial_r g_{11} + \partial_r g_{11} - \partial_r g_{11}]/2 = 0$\\
当$i=j=2$时, $\Gamma^1_{22}=g^{11}\qty[-\partial_r g_{22}]/2 = -r$\\
当$i\neq j$时, $\Gamma^1_{ij}=\Gamma^1_{ji}$, 取$i=1, j=2$, $\Gamma^1_{12} = g^{11}\qty[\partial_\theta g_{11}]/2 = 0$
\paragraph{}
同理计算$\Gamma^2_{ij}:$
\begin{align*}
    \Gamma^2_{ij} &= \frac{g^{2 k}}{2}\qty[\pdv{g_{ik}}{x^j} + \pdv{g_{jk}}{x^i} - \pdv{g_{ij}}{x^k}]\\
    & = \frac{g^{22}}{2}\qty[\pdv{g_{i2}}{x^j} + \pdv{g_{j2}}{x^i} - \pdv{g_{ij}}{x^2}]
\end{align*}
当$i=j=1$时, $\Gamma^2_{11}=g^{22}\qty[- \partial_\theta g_{22}]/2 = 0$\\
当$i=j=2$时, $\Gamma^2_{22}=g^{22}\qty[\partial_\theta g_{22}]/2 = 0$\\
当$i\neq j$时, $\Gamma^2_{ij}=\Gamma^2_{ji}$, 取$i=1, j=2$, $\Gamma^1_{22} = g^{22}\qty[\partial_r g_{22}]/2 = 1/r$

所以非零项只有$$\Gamma^2_{12}=\Gamma^2_{21}=\frac{1}{r};\quad \Gamma^1_{22} = -r$$
\paragraph{(b)}
将上面的克氏符带入到测地线方程\ref{geodesic}中, 得:
\begin{align*}
    \dv[2]{x^1}{t} + \Gamma^1_{22}\dv{x^2}{t}\dv{x^2}{t} &= 0\\
    \dv[2]{x^2}{t} + 2\times\Gamma^2_{12}\dv{x^1}{t}\dv{x^2}{t} &= 0\\
    \dv[2]{r}{t} -r\qty(\dv{\theta}{t})^2 &= 0\\
    \dv[2]{\theta}{t} +\frac1r \dv{\theta}{t} \dv{r}{t}&= 0 \label{haha}\\
\end{align*}
% \section{Exercise 3.}
% $g_{\mu\nu}$的形式为:
$$g_{\mu\nu} = \mqty(\dmat[0]{-1 - 2 \phi,1 - 2\phi,1- 2\phi,1- 2\phi})$$
\paragraph{(a)}
利用公式\ref{chris}得
\begin{align*}
    \Gamma^0_{00} &= \frac{g^{0\nu}}{2}\qty[\pdv{g_{0\nu}}{x^0}+\pdv{g_{0\nu}}{x^0}-\pdv{g_{00}}{x^\nu}]\\
        &=\frac{g^{00}}{2}\qty[\pdv{g_{00}}{x^0}] = \frac{1}{1 + 2\phi} \pdv{\phi}{t}\\
        &\sim\pdv{\phi}{t}
\end{align*}
\begin{align*}
    \Gamma^i_{00} &= \frac{g^{i\nu}}{2}\qty[\pdv{g_{0\nu}}{x^0}+\pdv{g_{0\nu}}{x^0}-\pdv{g_{00}}{x^\nu}]\\
    & = \frac{\delta^{ij} g^{11}}{2}\qty[\pdv{g_{0j}}{x^0}+\pdv{g_{0j}}{x^0}-\pdv{g_{00}}{x^j}] \\
    & = -\frac{\delta^{ij} g^{11}}{2}\pdv{g_{00}}{x^j} = \frac{1}{1-2\phi} \delta^{ij}\pdv{\phi}{x^j}\\
    & \sim \delta^{ij} \pdv{\phi}{x^j}
\end{align*}


\paragraph{(b)}
利用测地线方程\ref{4dgeodesic}, 在时间分量上有:
\begin{align*}
    \dv[2]{x^0}{\lambda} + \Gamma^0_{\alpha\beta}\dv{x^\alpha}{\lambda}\dv{x^\beta}{\lambda} = 0
\end{align*}
计算$\Gamma$
% \section{Exercise 4.}
% FRW度规\ref{FRWmertic}下的克氏符:
\begin{align*}
    &\Gamma^0_{00} = 0\\
    &\Gamma^0_{0i} = 0\\
    &\Gamma^0_{ij} = \delta_{ij}\dot{a} a \quad\text{(见书30页)}
\end{align*}
测地线方程\ref{4dgeodesic}的时间分量:
\begin{align*}
    \dv[2]{x^0}{\lambda} + \Gamma^0_{ij}\dv{x^i}{\lambda}\dv{x^j}{\lambda} &= 0\\
    E\dv{E}{t} + \dot{a} aP^iP^i &= 0\\
    g_{\mu\nu}P^\mu P^\nu &= -m^2\\
    -E^2 + a^2P^iP^i &= -m^2\\
    E\dv{E}{t} + \frac{\dot{a}}{a}(E^2 - m^2) &= 0\\
    \frac{\dd E^2}{2(E^2-m^2)} + \frac{\dd a}{a}&=0\\
    \frac{1}{2}\dd \ln(E^2-m^2) + \dd \ln a &= 0\\
    \dd \ln(a \sqrt{E^2-m^2}) &= 0\\
    \sqrt{E^2-m^2} &\propto \frac{1}{a}
\end{align*}

% \section{Exercise 5.}\label{sec5}
% \paragraph{(a)}
在FRW度规\ref{FRWmertic}下计算$\Gamma^i_{\alpha\beta}$:
\begin{align*}
    \Gamma^i_{\alpha\beta} = \frac{g^{il}}{2}\qty[\pdv{g_{\alpha l}}{x^\beta} + \pdv{g_{\beta l}}{x^\alpha} - \pdv{g_{\alpha\beta}}{x^l}]
\end{align*}

$\Gamma^i_{00}$:
\begin{align*}
    \Gamma^i_{00} &= \frac{g^{il}}{2}\qty[\pdv{g_{0 l}}{x^0} + \pdv{g_{0 l}}{x^0} - \pdv{g_{00}}{x^l}]\\
        &=g^{il}\pdv{g_{0 l}}{x^0} = 0
\end{align*}
$\Gamma^i_{0j} = \Gamma^i_{j0}$:
\begin{align*}
    \Gamma^i_{j0} &= \frac{g^{il}}{2}\qty[\pdv{g_{j l}}{x^0} + \pdv{g_{0 l}}{x^j} - \pdv{g_{j0}}{x^l}]\\
    &=\frac{g^{il}}{2} \pdv{g_{j l}}{x^0} = \delta_{ij}\frac{\dot{a}}{a}
\end{align*}
$\Gamma^i_{jk}$:
\begin{align*}
    \Gamma^i_{jk} &= \frac{g^{il}}{2}\qty[\pdv{g_{j l}}{x^k} + \pdv{g_{k l}}{x^j} - \pdv{g_{jk}}{x^l}]\\
        &=\frac{1}{2a^2}\delta_{il}\qty[\delta_{j l}\pdv{a^2}{x^k} + \delta_{k l}\pdv{a^2}{x^j} - \delta_{jk}\pdv{a^2}{x^l}]\\
        &=\frac{1}{a}\qty[\delta_{ij}\pdv{a}{x^k} + \delta_{ik}\pdv{a}{x^j} - \delta_{jk}\pdv{a}{x^i}]\\
        &=0
\end{align*}

\paragraph{(b)}
\begin{align*}
    R_{ij} =& \Gamma^\alpha_{ij, \alpha} - \Gamma^\alpha_{i\alpha, j} + \Gamma^\alpha_{\beta\alpha}\Gamma^\beta_{ij} - \Gamma^\alpha_{\beta j}\Gamma^\beta_{i \alpha} \\
        =&\qty(\Gamma^0_{ij, 0} + \Gamma^k_{ij, k}) - \qty(\Gamma^0_{i0, j} + \Gamma^k_{ik, j}) + \qty(\Gamma^0_{\beta 0} + \Gamma^k_{\beta k})\Gamma^\beta_{ij} - \qty(\Gamma^0_{\beta j}\Gamma^\beta_{i 0} + \Gamma^k_{\beta j}\Gamma^\beta_{i k})\\
        =&\qty(\Gamma^0_{ij, 0} + \Gamma^k_{ij, k}) - \qty(\Gamma^0_{i0, j} + \Gamma^k_{ik, j}) + \qty(\Gamma^0_{0 0} + \Gamma^k_{0 k})\Gamma^0_{ij} + \qty(\Gamma^0_{n 0} + \Gamma^k_{n k})\Gamma^n_{ij} - \\
         &\qty(\Gamma^0_{0 j}\Gamma^0_{i 0} + \Gamma^0_{n j}\Gamma^n_{i 0} + \Gamma^k_{0 j}\Gamma^0_{i k} + \Gamma^k_{n j}\Gamma^n_{i k})\\
        =&\qty[\partial_0\qty(\delta_{ij}\dot{a}a) + 0] - \qty(0+0) + \qty(0+\delta_{kk}\frac{\dot{a}}{a})\delta_{ij}\dot{a}a + \qty(0+0)\times 0 - \\
         &\qty(0\times0+\delta_{nj}\dot{a}a\delta_{ni}\frac{\dot{a}}{a} + \delta_{kj}\frac{\dot{a}}{a}\delta_{ik}\dot{a}a + 0)\\
        =&\delta_{ij} \qty(\ddot{a}a + \dot{a}^2) + 3\delta_{ij}\dot{a}^2 - 2\delta_{ij}\dot{a}^2\\
        =&\delta_{ij} \qty(\ddot{a}a + 2\dot{a}^2)
\end{align*}
\begin{align*}
    R_{i0} =&\qty(\Gamma^0_{i0, 0} + \Gamma^k_{i0, k}) - \qty(\Gamma^0_{i0, 0} + \Gamma^k_{ik, 0}) + \qty(\Gamma^0_{0 0} + \Gamma^k_{0 k})\Gamma^0_{i0} + \qty(\Gamma^0_{n 0} + \Gamma^k_{n k})\Gamma^n_{i0} - \\
        &\qty(\Gamma^0_{0 0}\Gamma^0_{i 0} + \Gamma^0_{n 0}\Gamma^n_{i 0} + \Gamma^k_{0 0}\Gamma^0_{i k} + \Gamma^k_{n 0}\Gamma^n_{i k})\\
        =& \qty[0+\delta_{ki}\partial_k\qty(\frac{\dot{a}}{a})] - \qty(0 + 0) + \qty(0 + \Gamma^k_{ik, 0}) \times 0 + \qty(0 + 0)\times\Gamma^n_{i0} - \\
        &\qty(0\times0 + 0 \times \Gamma^n_{i 0} + 0 \times \Gamma^0_{i k} + \Gamma^k_{n 0} \times 0)
        = 0
\end{align*}
% \section{Exercise 6.}
% 由爱因斯坦场方程\ref{einsteineq} 及理想各向同性流体假设, 有:
\begin{align*}
    R_{\mu\nu} - \frac{1}{2}g_{\mu\nu}\mathcal{R} = 8\pi G g_{\mu\tau}T^{\tau}_{\ \nu}\\
    T^\tau_{\ \nu} = \mqty(\dmat[0]{-\rho, \mathcal{P}, \mathcal{P}, \mathcal{P}})
\end{align*}
Ricci 标量
\begin{align*}
    \mathcal{R} &= R_{\mu\nu}g^{\mu\nu}\\
    &=-R_{00} + \frac{1}{a^2}R_{ii}\\
    &=3\frac{\ddot{a}}{a} + \frac{3}{a^2}\qty(2\dot{a}^2 + a\ddot{a})\\
    &=6\qty[\frac{\ddot{a}}{a} + \qty(\frac{\dot{a}}{a})^2]
\end{align*}
空间分量:
\begin{align*}
    R_{ij} - \frac{1}{2}g_{ij}\mathcal{R} &= 8\pi G g_{i\tau}T^{\tau}_{\ j}\\
    &= 8\pi G a^2\delta_{ik}\delta_{kj}\mathcal{P}\\
    &= 8\pi G a^2\mathcal{P}\delta_{ij}
\end{align*}
当$i=j=1,2,3$时给出同一个方程:
\begin{align*}
    R_{11} - \frac{1}{2}g_{11}\mathcal{R} &= 8\pi G a^2 \mathcal{P}\\
    2\dot{a}^2 + a\ddot{a} - 3a^2 \qty[\frac{\ddot{a}}{a} + \qty(\frac{\dot{a}}{a})^2] &= 8\pi G a^2 \mathcal{P}\\
    2\dot{a}^2 + a\ddot{a} - 3a\ddot{a} - 3\dot{a}^2 &= 8\pi G a^2 \mathcal{P}\\
    \dot{a}^2 + 2a\ddot{a} &= -8\pi G a^2 \mathcal{P}\\
    \frac{\ddot{a}}{a} + \frac12\qty(\frac{\dot{a}}{a})^2 &= -4\pi G \mathcal{P}
\end{align*}
% \section{Exercise 7.}
% \paragraph{(a)}
度规$g_{\mu\nu}$:
$$g_{\mu\nu} = \mqty(\dmat[0]{-1, r^2, r^2\sin^2\theta})$$
坐标$x^\mu = \qty{t, \theta, \phi}, \mu = 1,2,3$\\
利用\ref{chris}:
\begin{align*}
    \Gamma^\mu_{\alpha\beta} &= \frac{g^{\mu\nu}}{2}\qty[\pdv{g_{\alpha\nu}}{x^\beta} + \pdv{g_{\beta\nu}}{x^\alpha} - \pdv{g_{\alpha\beta}}{x^\nu}]\\
    &=\frac{\delta^{\mu\nu}i(\mu)}{2}\qty[\delta_{\alpha\nu}\pdv{d(\alpha)}{x^\beta} + \delta_{\beta\nu}\pdv{d(\beta)}{x^\alpha} - \delta_{\alpha\beta}\pdv{d(\alpha)}{x^\nu}]\\
    &=\frac{i(\mu)}{2}\qty[\delta_\alpha^\mu\pdv{d(\mu)}{x^\beta} + \delta_\beta^\mu\pdv{d(\mu)}{x^\alpha} - \delta_{\alpha\beta}\pdv{d(\alpha)}{x^\mu}]
\end{align*}
其中$d(\mu) = \qty{-1, r^2, r^2\sin^2\theta}, i(\mu) = \qty{-1, {1}/{r^2}, {1}/(r^2\sin^2\theta)} \mu = 1, 2, 3$\\
$\Gamma^1_{\alpha\beta}:$
\begin{align*}
    \Gamma^1_{\alpha\beta} &=\frac{i(1)}{2}\qty[\delta_\alpha^1\pdv{d(1)}{x^\beta} + \delta_\beta^1\pdv{d(1)}{x^\alpha} - \delta_{\alpha\beta}\pdv{d(\alpha)}{x^1}]\\
    &=0
\end{align*}

$\Gamma^2_{\alpha\beta}:$
\begin{align*}
    \Gamma^2_{\alpha\beta} &=\frac{i(2)}{2}\qty[\delta_\alpha^2\pdv{d(2)}{x^\beta} + \delta_\beta^2\pdv{d(2)}{x^\alpha} - \delta_{\alpha\beta}\pdv{d(\alpha)}{x^2}]\\
    &=-\frac{i(2)}{2}\delta_{\alpha\beta}\pdv{d(\alpha)}{x^2}\\
    \Gamma^2_{33} &= -\frac{1}{2r^2}\pdv{r^2\sin^2\theta}{\theta} = -\sin\theta\cos\theta
\end{align*}

$\Gamma^3_{\alpha\beta}:$
\begin{align*}
    \Gamma^3_{\alpha\beta} &=\frac{i(3)}{2}\qty[\delta_\alpha^3\pdv{d(3)}{x^\beta} + \delta_\beta^3\pdv{d(3)}{x^\alpha} - \delta_{\alpha\beta}\pdv{d(\alpha)}{x^3}]\\
    &=\frac{1}{2r^2\sin^2\theta}\qty[\delta_\alpha^3\pdv{r^2\sin^2\theta}{x^\beta} + \delta_\beta^3\pdv{r^2\sin^2\theta}{x^\alpha}]\\
    \Gamma^3_{32} = \Gamma^3_{23} &=\frac{1}{2r^2\sin^2\theta}\pdv{r^2\sin^2\theta}{x^2}\\
    & =\frac{\cos\theta}{\sin\theta}
\end{align*}
\paragraph{(b)}
由测地线方程\ref{4dgeodesic}:
\begin{align*}
        \dv[2]{x^2}{t} + \Gamma^2_{jk}\dv{x^j}{t}\dv{x^k}{t} &= 0\\
        \dv[2]{\theta}{t} + \Gamma^2_{33}\dv{\phi}{t}\dv{\phi}{t} &= 0\\
        \ddot{\theta} - \sin\theta\cos\theta\dot{\phi}^2 &=0
\end{align*}
 \begin{align*}
        \dv[2]{x^3}{t} + \Gamma^3_{jk}\dv{x^j}{t}\dv{x^k}{t} &= 0\\
        \dv[2]{\phi}{t} + 2\Gamma^3_{23}\dv{\theta}{t}\dv{\phi}{t} &= 0\\
        \ddot{\phi} + 2\frac{\cos\theta}{\sin\theta}\dot{\theta}\dot{\phi} &=0
\end{align*}
\paragraph{(c)}
Ricci 张量\ref{riccit}:
\begin{align*}
    R_{\mu\nu} &= \Gamma^\alpha_{\mu\nu, \alpha} - \Gamma^\alpha_{\mu\alpha, \nu} + \Gamma^\alpha_{\beta\alpha}\Gamma^\beta_{\mu\nu} - \Gamma^\alpha_{\beta\nu}\Gamma^\beta_{\mu\alpha}\\
    &=\Gamma^2_{\mu\nu,2} + \Gamma^3_{\mu\nu,3} - \Gamma^3_{\mu3, \nu} + \Gamma^3_{\beta 3}\Gamma^\beta_{\mu\nu} - \qty(\Gamma^2_{\beta\nu}\Gamma^\beta_{\mu 2} + \Gamma^3_{\beta\nu}\Gamma^\beta_{\mu 3})\\
    &=\Gamma^2_{\mu\nu,2} + \Gamma^3_{\mu\nu,3} - \Gamma^3_{\mu3, \nu} +\Gamma^3_{2 3}\Gamma^2_{\mu\nu} - \qty(\Gamma^2_{3\nu}\Gamma^3_{\mu 2} + \Gamma^3_{2\nu}\Gamma^2_{\mu 3} + \Gamma^3_{3\nu}\Gamma^3_{\mu 3})\\
\end{align*}
\begin{align*}
    R_{22} &= 0 + 0 - \Gamma^3_{23, 2} - \qty(0 + 0 + \Gamma^3_{32}\Gamma^3_{23})\\
        &=-\partial_\theta\frac{\cos\theta}{\sin\theta} - \qty(\frac{\cos\theta}{\sin\theta})^2\\
        &=\frac{1}{\sin^2\theta}-\frac{\cos^2\theta}{\sin^2\theta} \\
        &=1\\
    \\
    R_{23} &= 0 + \Gamma^3_{23,3}-\Gamma^3_{23,3} + 0 - \qty(0 + 0 + 0)\\
        &= 0\\
    \\
    R_{32} &= 0 + \Gamma^3_{32,3} - 0 + 0 - \qty(0 + 0 + 0)\\
    &=0\\
    \\
    R_{33} &= \Gamma^2_{33,2} + 0 - 0 + \Gamma^3_{23}\Gamma^2_{33} - \qty(\Gamma^2_{33}\Gamma^3_{32}+\Gamma^3_{23}\Gamma^2_{33}+0)\\ 
        & = -\partial_\theta\qty(\sin\theta\cos\theta) + \cos^2\theta \\
        & = \sin^2\theta
\end{align*}
其余均为$0$
\begin{align*}
    \mathcal{R} &= g^{\mu\nu}R_{\mu\nu}\\
        & = g^{22}R_{22} + g^{33}R_{33}\\
        & = \frac{1}{r^2} + \frac{1}{r^2\sin^2\theta}\sin^2\theta\\
        & = \frac{2}{r^2} 
\end{align*}
% \section{Exercise 8.}
% 开放宇宙线元:
\begin{equation*}
    \dd s^2 = -\dd t^2 + a^2(t)\qty{\frac{\dd r^2}{1+\Omega_kH_0^2r^2} + r^2\qty(\dd \theta^2 + \sin^2 \theta \dd \phi^2)}
\end{equation*}
其度规为:
\begin{align*}
    g_{\mu\nu} = \mqty(\dmat[0]{-1, \frac{a^2(t)}{1 + \Omega_kH_0^2r^2}, r^2a^2(t), r^2\sin^2\theta a^2(t)})
\end{align*}
\paragraph{(a)}
由\ref{chris}:
\begin{align*}
    \Gamma^\mu_{\alpha\beta} &= \frac{g^{\mu\nu}}{2}\qty[\pdv{g_{\alpha\nu}}{x^\beta} + \pdv{g_{\beta\nu}}{x^\alpha} - \pdv{g_{\alpha\beta}}{x^\nu}]\\
\end{align*}
$\Gamma^0_{\alpha\beta}$
\begin{align*}
    \Gamma^0_{\alpha\beta} &= \frac{g^{00}}{2}\qty[\pdv{g_{\alpha 0}}{x^\beta} + \pdv{g_{\beta 0}}{x^\alpha} - \pdv{g_{\alpha\beta}}{x^0}]\\        
\end{align*}
因为$\Gamma^0_{00} = 0$:
\begin{align*}
    \Gamma^0_{ij} &= -\frac{g^{00}}{2}\pdv{g_{ij}}{t} \\
    &=\frac{1}{2}g_{ij}\times \frac{2}{a}\dv{a}{t} \\
    &=g_{ij}H
\end{align*}
计算$\Gamma^i_{\alpha 0} =\Gamma^i_{0\alpha} $
\begin{align*}
    \Gamma^i_{\alpha 0} &= \frac{g^{i\nu}}{2}\qty[\pdv{g_{\alpha \nu}}{x^0} + \pdv{g_{0 \nu}}{x^\alpha} - \pdv{g_{\alpha 0}}{x^\nu}]\\        
        &= \frac{g^{ik}}{2}\pdv{g_{\alpha k}}{t}
\end{align*}
非零项为:
\begin{align*}
    \Gamma^i_{j0} &=\frac{g^{ik}}{2}\pdv{g_{jk}}{t}\\
    & = \frac{g^{ik}}{2}g_{jk}\frac{2a\dd a/\dd t}{a^2}\\
    & = \delta^i_jH
\end{align*}
计算$\Gamma^i_{jk}$:
\begin{align*}
    \Gamma^i_{jk} &= \frac{g^{i\nu}}{2}\qty[\pdv{g_{j\nu}}{x^k} + \pdv{g_{k\nu}}{x^j} - \pdv{g_{jk}}{x^\nu}]\\
        &=\frac{g^{il}}{2}\qty[\pdv{g_{jl}}{x^k} + \pdv{g_{kl}}{x^j} - \pdv{g_{jk}}{x^l}]\\
        &=\frac{I(i)}{2}\qty[g_{ji,k} + g_{ki, j} - g_{jk, i}]
\end{align*}
\begin{align*}
    \Gamma^1_{11} &= \frac{g^{11}}{2}g_{11,1} = -\frac{\Omega_kH_0^2r}{1 +\Omega_kH_0^2r^2}\\
    \Gamma^1_{22} &= -\frac{g^{11}}{2}g_{22,1} = -r\qty(1 + \Omega_kH_0^2r^2)\\
    \Gamma^1_{33} &= -\frac{g^{11}}{2}g_{33,1} = -r\sin^2\theta\qty(1 + \Omega_kH_0^2r^2)\\\
    \Gamma^1_{12} &= \Gamma^1_{21} = \frac{g^{11}}{2}g_{11,2} = 0\\
    \Gamma^1_{13} &= \Gamma^1_{31} = \frac{g^{11}}{2}g_{11,3} = 0\\
    \Gamma^1_{23} &= \Gamma^1_{32} = 0 \\
    \Gamma^2_{11} &= -\frac{g^{22}}{2}g_{11,2} = 0\\
    \Gamma^2_{22} &= \frac{g^{22}}{2}g_{22,2} = 0\\
    \Gamma^2_{33} &= -\frac{g^{22}}{2}g_{33,2} = -\sin\theta\cos\theta\\
    \Gamma^2_{12} &= \Gamma^2_{21} = \frac{g^{22}}{2}g_{22,1} = \frac{1}{r} \\
    \Gamma^2_{13} &= \Gamma^2_{31} = 0 \\
    \Gamma^2_{23} &= \Gamma^2_{32} = 0 \\
    \Gamma^3_{11} &= -\frac{g^{33}}{2}g_{11,3} = 0 \\
    \Gamma^3_{22} &= -\frac{g^{33}}{2}g_{22,3} = 0 \\
    \Gamma^3_{33} &= \frac{g^{33}}{2}g_{33,3} = 0 \\
    \Gamma^3_{12} &= \Gamma^3_{21} = 0 \\
    \Gamma^3_{13} &= \Gamma^3_{31} = \frac{g^{33}}{2}g_{33,1} = \frac{1}{r}\\
    \Gamma^3_{23} &= \Gamma^3_{32} = \frac{g^{33}}{2}g_{33,2} = \frac{\cos\theta}{\sin\theta}\\
\end{align*}


\paragraph{(b)}
由Ricci张量的定义\ref{riccit}:
\begin{align*}
    R_{\mu\nu} &= \Gamma^\alpha_{\mu\nu, \alpha} - \Gamma^\alpha_{\mu\alpha, \nu} + \Gamma^\alpha_{\beta\alpha}\Gamma^\beta_{\mu\nu} - \Gamma^\alpha_{\beta\nu}\Gamma^\beta_{\mu\alpha}\\
\end{align*}
\begin{align*}
    R_{00} &= \Gamma^\alpha_{00, \alpha} - \Gamma^\alpha_{0\alpha, 0} + \Gamma^\alpha_{\beta\alpha}\Gamma^\beta_{00} - \Gamma^\alpha_{\beta 0}\Gamma^\beta_{0\alpha}\\
        & = 0 - \Gamma^i_{0i,0} + 0 - \Gamma^i_{j0}\Gamma^j_{0i}\\
        & = -\delta^i_i\partial_tH - \delta^i_j\delta^j_iH^2 \\
        & = -3\qty(\dot{H} + H^2)\\
        & = -3\qty[\dv{t}(\frac{\dot{a}}{a}) + \qty(\frac{\dot{a}}{a})^2]\\
        & = -3\qty[\frac{a\ddot{a} - \dot{a}^2}{a^2} +\qty(\frac{\dot{a}}{a})^2]\\
        & = -3 \frac{\ddot{a}}{a}
\end{align*}
\begin{align*}
    R_{ij} &= \Gamma^\alpha_{ij, \alpha} - \Gamma^\alpha_{i\alpha, j} + \Gamma^\alpha_{\beta\alpha}\Gamma^\beta_{ij} - \Gamma^\alpha_{\beta j}\Gamma^\beta_{i\alpha}\\
        & = \qty(\Gamma^0_{ij, 0} + \Gamma^k_{ij, k}) - \Gamma^k_{ik,j} + \Gamma^k_{\beta k}\Gamma^\beta_{ij} - \qty(\Gamma^0_{\beta j}\Gamma^\beta_{i0} + \Gamma^k_{\beta j}\Gamma^\beta_{ik}) \\
        & = \qty(\Gamma^0_{ij, 0} + \Gamma^k_{ij, k}) - \Gamma^k_{ik,j} + \qty(\Gamma^k_{0 k}\Gamma^0_{ij} + \Gamma^k_{l k}\Gamma^l_{ij}) - \qty(\Gamma^0_{l j}\Gamma^l_{i0} + \Gamma^k_{0 j}\Gamma^0_{ik} + \Gamma^k_{l j}\Gamma^l_{ik}) \\
        & = \partial_t\qty(g_{ij}H) + \Gamma^k_{ij, k} - \Gamma^k_{ik,j} + \delta^k_kg_{ij}H^2 + \Gamma^k_{lk}\Gamma^l_{ij} - g_{lj}\delta^l_iH^2 - \delta^k_jg_{ik}H^2 - \Gamma^k_{lj}\Gamma^l_{ik}\\
        & = g_{ij}\qty(2H^2 + \dot{H}) + \Gamma^k_{ij, k} - \Gamma^k_{ik,j} + 3g_{ij}H^2 + \Gamma^k_{lk}\Gamma^l_{ij} - 2g_{ij}H^2 -  \Gamma^k_{lj}\Gamma^l_{ik}\\
        & = g_{ij}\qty(3H^2 + \dot{H}) + \Gamma^k_{ij, k} - \Gamma^k_{ik,j}  + \Gamma^k_{lk}\Gamma^l_{ij}  -  \Gamma^k_{lj}\Gamma^l_{ik} \\
        & = g_{ij}\qty[2H^2 + \frac{\ddot{a}}{a}]+ \Gamma^k_{ij, k} - \Gamma^k_{ik,j}  + \Gamma^k_{lk}\Gamma^l_{ij}  -  \Gamma^k_{lj}\Gamma^l_{ik} \\
\end{align*}
每项的计算:\\
(1)\\
\begin{align*}
    \Gamma^k_{ij,k} =& \Gamma^1_{ij, 1} + \Gamma^2_{ij, 2} + \Gamma^3_{ij, 3}\\
        =&\mqty(\Gamma^1_{11,1} & 0 & 0 \\ 0 & \Gamma^1_{22,1} & 0 \\ 0 & 0 & \Gamma^1_{33,1} + \Gamma^2_{33,2})\\
        % =&\partial_r \mqty(\dmat[0]{-\frac{\Omega_kH_0^2r}{1 +\Omega_kH_0^2r^2}, -r\qty(1 + \Omega_kH_0^2r^2),  -r\sin^2\theta\qty(1 + \Omega_kH_0^2r^2)})\\
        % &+\partial_\theta\mqty(0&1/r&0\\1/r&0&0 \\0&0&-\sin\theta\cos\theta) + 0 \\
\end{align*}
% \begin{align*}
%     \Gamma^k_{11,k} &= \partial_1\partial_1\ln g_{11} - \frac{1}{2}\partial_l\partial_lg_{11} =\frac{1}{2}\partial_1\partial_1\ln g_{11}\\
%     \Gamma^k_{22,k} &= \partial_2\partial_2\ln g_{22} - \frac{1}{2}\partial_l\partial_lg_{22} = -\frac{1}{2}\partial_1\partial_1\ln g_{22}\\
%     \Gamma^k_{33,k} &= \partial_3\partial_3\ln g_{33} - \frac{1}{2}\partial_l\partial_lg_{33} = -\frac12\qty(\partial_1\partial_1g_{33}+\partial_2\partial_2g_{33})\\
%     \Gamma^k_{12,k} &= \Gamma^k_{21,k}=\Gamma^k_{13,k}=\Gamma^k_{31,k}=\Gamma^k_{23,k}=\Gamma^k_{32,k} = 0
% \end{align*}
(2)
\begin{align*}
    \Gamma^k_{ik,j} &= \Gamma^1_{i1,j} + \Gamma^2_{i2,j} + \Gamma^3_{i3,j}\\
    &=\mqty(\Gamma^1_{11,1} + \Gamma^2_{12,1} + \Gamma^3_{13,1} & 0 & 0 \\ 0 & \Gamma^3_{23,2} & 0\\ 0 & 0 & 0)
\end{align*}
% \begin{align*}
%     \Gamma^k_{1k,1} &= \frac{1}{2}\partial_1\partial_1\ln\qty(g_{11}g_{22}g_{33})\\
%     \Gamma^k_{2k,2} &= \frac{1}{2}\partial_2\partial_2\ln\qty(g_{11}g_{22}g_{33})\\
%     \Gamma^k_{3k,3} &= \frac{1}{2}\partial_3\partial_3\ln\qty(g_{11}g_{22}g_{33}) = 0\\
%     \Gamma^k_{1k,2} &= \Gamma^k_{2k,1} = \frac{1}{2}\partial_1\partial_2\ln\qty(g_{11}g_{22}g_{33}) \\
%     \Gamma^k_{1k,3} &= \Gamma^k_{3k,1} = \Gamma^k_{2k,3} = \Gamma^k_{3k,2} = 0
% \end{align*}
(3)
\begin{align*}
    \Gamma^k_{lk}\Gamma^l_{ij} &= \Gamma^1_{ij}\qty(\Gamma^1_{11} + \Gamma^2_{12} + \Gamma^3_{13}) + \Gamma^2_{ij}\qty(\Gamma^1_{21} + \Gamma^2_{22} + \Gamma^3_{23}) + \Gamma^3_{ij}\qty(\Gamma^1_{31} + \Gamma^2_{32} + \Gamma^3_{33}) \\
        & = \Gamma^1_{ij}\qty(\Gamma^1_{11} + \Gamma^2_{12} + \Gamma^3_{13}) + \Gamma^2_{ij}\Gamma^3_{23}\\
        & = \mqty(\Gamma^1_{11}\qty(\Gamma^1_{11} + \Gamma^2_{12} + \Gamma^3_{13}) & \Gamma^2_{12}\Gamma^3_{23} & 0 \\ 
                    \Gamma^2_{21}\Gamma^3_{23} & \Gamma^1_{22}\qty(\Gamma^1_{11} + \Gamma^2_{12} + \Gamma^3_{13}) & 0\\
                    0 & 0 & \Gamma^1_{33}\qty(\Gamma^1_{11} + \Gamma^2_{12} + \Gamma^3_{13}) + \Gamma^2_{33}\Gamma^3_{23})
\end{align*}
% \begin{align*}
%     \Gamma^k_{lk}\Gamma^l_{11} &= \frac{1}{4}\partial_1\ln(g_{11}g_{22}g_{33})\partial_1\ln g_{11}\\
%     \Gamma^k_{lk}\Gamma^l_{22} &= \frac{1}{4}\qty{-I(1)\partial_1\ln \qty(g_{11}g_{22}g_{33})\partial_1g_{22}}\\
%     \Gamma^k_{lk}\Gamma^l_{33} &= \frac{1}{4}\partial_1\ln(g_{11}g_{22}g_{33})\partial_1\ln g_{11}
% \end{align*}
(4)
\begin{align*}
    \Gamma^k_{lj}\Gamma^l_{ik} &= \Gamma^1_{1j}\Gamma^1_{i1} + \Gamma^2_{1j}\Gamma^1_{i2} + \Gamma^3_{1j}\Gamma^1_{i3} + \Gamma^1_{2j}\Gamma^2_{i1} + \Gamma^2_{2j}\Gamma^2_{i2} + \Gamma^3_{2j}\Gamma^2_{i3} + \Gamma^1_{3j}\Gamma^3_{i1} + \Gamma^2_{3j}\Gamma^3_{i2} + \Gamma^3_{3j}\Gamma^3_{i3} \\
    &=\mqty(\Gamma^1_{11}\Gamma^1_{11} + \Gamma^2_{21}\Gamma^2_{12} + \Gamma^3_{31}\Gamma^3_{13} & \Gamma^3_{32}\Gamma^3_{13} & 0\\
        \Gamma^3_{31}\Gamma^3_{23} & 2\Gamma^1_{22}\Gamma^2_{21} + \Gamma^3_{32}\Gamma^3_{23} & 0\\
        0 & 0 & \Gamma^3_{23}\Gamma^2_{33} + 2\Gamma^1_{33}\Gamma^3_{31} + \Gamma^2_{33}\Gamma^3_{32})
\end{align*}
\begin{align*}
    \Gamma^k_{ij,k} - \Gamma^k_{ik,j} &= \mqty(-\Gamma^2_{12,1}-\Gamma^3_{13,1} & 0 & 0\\
        0 & \Gamma^1_{22,1} - \Gamma^3_{23,2} & 0 \\
        0 & 0 & \Gamma^1_{33,1} + \Gamma^2_{33,2}) \\
        & =\mqty(\frac{2}{r^2} & 0 & 0 \\ 0 & -\qty(1 + 3\Omega_kH_0^2r^2) + 1/\sin^2\theta & 0 \\ 0 & 0 & -\sin^2\theta\qty(1 + 3\Omega_kH_0^2r^2) - \cos 2\theta ) \\
        & =\mqty(\frac{2}{r^2} & 0 & 0 \\ 0 &  - 3\Omega_kH_0^2r^2 + \frac{\cos^2\theta}{\sin^2\theta} & 0 \\ 0 & 0 &  -3\Omega_kH_0^2r^2\sin^2\theta - \cos^2\theta ) 
\end{align*}
\begin{align*}
    &\Gamma^k_{lk}\Gamma^l_{ij}  -  \Gamma^k_{lj}\Gamma^l_{ik} \\
        =& \mqty(\Gamma^1_{11}\qty(\Gamma^2_{12} + \Gamma^3_{13}) - \Gamma^2_{12}\Gamma^2_{21} - \Gamma^3_{13}\Gamma^3_{31}   & \Gamma^3_{32}\qty(\Gamma^2_{12} - \Gamma^3_{13})& 0 \\
        \Gamma^3_{23}\qty(\Gamma^2_{21} - \Gamma^3_{31}) & \Gamma^1_{22}\qty(\Gamma^1_{11}-\Gamma^2_{12}+\Gamma^3_{13}) - \Gamma^3_{32}\Gamma^3_{23}& 0 \\ 
        0 & 0 & \Gamma^1_{33}\qty(\Gamma^1_{11}+\Gamma^2_{12} - \Gamma^3_{31}) - \Gamma^3_{23}\Gamma^2_{33}) \\
        =&\mqty(-\frac{2\Omega_kH_0^2}{1+\Omega_kH_0^2r^2} - \frac{2}{r^2} & 0 & 0\\ 0& \Omega_kH_0^2r^2 - \frac{\cos^2\theta}{\sin^2\theta}&0\\0&0& \Omega_kH_0^2\sin^2\theta r^2 + \cos^2\theta ) 
\end{align*}
\begin{align*}
    &\Gamma^k_{ij, k} - \Gamma^k_{ik,j}  + \Gamma^k_{lk}\Gamma^l_{ij}  -  \Gamma^k_{lj}\Gamma^l_{ik} \\
    =&\mqty(-\frac{2\Omega_kH_0^2}{1+\Omega_kH_0^2r^2}  & 0 & 0\\ 0& -2\Omega_kH_0^2r^2 &0\\0&0& -2\Omega_kH_0^2\sin^2\theta r^2 ) \\
    =&-\frac{2\Omega_kH_0^2}{a^2}g_{ij}
\intertext{最终}
    R_{ij} =& g_{ij}\qty(2H^2+\frac{\ddot{a}}{a} - \frac{2\Omega_kH_0^2}{a^2})
\end{align*}
\paragraph{(c)}
Ricci标量:
\begin{align*}
    \mathcal{R} &= g^{\mu\nu}R_{\mu\nu} = g^{00}R_{00} + g^{11}R_{11} +g^{22}R_{22} +g^{33}R_{33}\\
        &=-R_{00} + \qty(g^{11}g_{11} + g^{22}g_{22} + g^{33}g_{33})  \qty(2H^2+\frac{\ddot{a}}{a} - \frac{2\Omega_kH_0^2}{a^2})\\
        &=3\frac{\ddot{a}}{a} + 3\qty(2H^2+\frac{\ddot{a}}{a} - \frac{2\Omega_kH_0^2}{a^2})\\
        &=6\qty(H^2+\frac{\ddot{a}}{a} - \frac{\Omega_kH_0^2}{a^2})
\end{align*}
爱因斯坦场方程\ref{einsteineq}:
\begin{align*}
    G_{\mu\nu}\equiv R_{\mu\nu} - \frac{1}{2}g_{\mu\nu}\mathcal{R} = 8\pi GT_{\mu\nu}\\
\end{align*}
时间分量:
\begin{align*}
    R_{00} + \frac{1}{2}\mathcal{R} &= 8\pi G T_{00}\\
    -3\frac{\ddot{a}}{a} + 3\qty(H^2+\frac{\ddot{a}}{a} - \frac{\Omega_kH_0^2}{a^2}) &= 8\pi G T^\mu_0g_{\mu 0}\\
    3\qty(H^2 - \frac{\Omega_kH_0^2}{a^2}) &= 8\pi G \rho \\
    H^2 - \frac{\Omega_kH_0^2}{a^2} &= \frac{8\pi G }{3}\rho\\
\end{align*}
空间分量
\begin{align*}
    R_{ii} - \frac{1}{2}g_{ii}\mathcal{R} &= 8 \pi G T^\tau_ig_{\tau i} &\text{$i$ 不求和} \\
    g_{ii}\qty(2H^2+\frac{\ddot{a}}{a} - \frac{2\Omega_kH_0^2}{a^2}) - 3g_{ii}\qty(H^2+\frac{\ddot{a}}{a} - \frac{\Omega_kH_0^2}{a^2}) &= 8\pi G \mathcal{P} g_{ii}\\
    \frac{\Omega_kH_0^2}{a^2} - H^2 - 2 \frac{\ddot{a}}{a} &= 8\pi G\mathcal{P} \\
    -2\frac{\ddot{a}}{a} &= 8\pi G \qty(\mathcal{P} + \frac{\rho}{3})\\
    \frac{\ddot{a}}{a} &= -\frac{4\pi G}{3}\qty(3\mathcal{P} + \rho)
\end{align*}

% \section{Exercise 9.}
% 由测地线方程\eqref{4dgeodesic} 和\ref{sec5}有:
\begin{align*}
    \dv[2]{x^i}{\lambda} + \Gamma^i_{\alpha\beta}\dv{x^\alpha}{\lambda}\dv{x^\beta}{\lambda} &= 0\\
    \dv[2]{x^i}{\lambda} + 2\Gamma^i_{j0}\dv{x^j}{\lambda}\dv{x^0}{\lambda} &= 0 \\
    \dv[2]{x^i}{\lambda} + 2\delta_{ij}\frac{\dot{a}}{a}\dv{x^j}{\lambda}\dv{x^0}{\lambda} &= 0\\
    \dv[2]{x^i}{\lambda} + 2\frac{\dot{a}}{a}\dv{x^i}{\lambda}\dv{x^0}{\lambda} &= 0 \\
    E\dv{t}(E\dv{x^i}{t}) + 2\frac{\dot{a}}{a}E^2\dv{x^i}{t} &= 0 \\
    \dv{E}{t}\dv{x^i}{t} + E\dv[2]{x^i}{t} + 2\frac{\dot{a}}{a}E\dv{x^i}{t} &= 0 \\
    \dv{E}{t}\dv{x^i}{t} + E\dv[2]{x^i}{t} + 2\frac{\dot{a}}{a}E\dv{x^i}{t} &= 0   &&\text{书31页$\dv{E}{t} + \frac{\dot{a}}{a}E = 0$}\\
    \dv[2]{x^i}{t} + \frac{\dot{a}}{a}\dv{x^i}{t} &= 0 \\
    \dv{t}(a\dv{x^i}{t}) & = 0\\
    \dv[2]{x^i}{\eta} &= 0 &&\text{$a\dd \eta = \dd t$}
\end{align*}
% \section{Exercise 10.}
% \begin{align*}
    \frac{H}{H_0} &= \sqrt{\frac{\rho}{\rho_{cr}}} = \sqrt{\frac{\rho_m + \rho_r}{\rho_{cr}}} \\
        &=\sqrt{\frac{\Omega_m}{a^3} + \frac{\Omega_r}{a^4}}\\
\intertext{因为$\Omega_r = 4.15\times 10 ^{-5}/h^2, h\sim0.72$, 且已经假设宇宙中只有物质和辐射, 所以$\Omega_m\sim1, \Omega_r = 8.005 \times 10^{-5}$}
    H(a) &= H_0\sqrt{\frac{1}{a^3} + \frac{\Omega_r}{a^4}}\\
    \dv{a}{t} &= aH_0\sqrt{\frac{1}{a^3} + \frac{\Omega_r}{a^4}}\\ 
        &=H_0\sqrt{\frac{1}{a} + \frac{\Omega_r}{a^2}}\\
    t(a)&= \frac{1}{H_0}\int_0^a \frac{a'\dd a'}{\sqrt{a' + \Omega_r}} =\frac{2}{H_0}\int_0^aa'\dd \sqrt{a' + \Omega_r} \\ 
        &=\frac{2}{H_0}\qty[a\sqrt{a + \Omega_r} - \int_0^a\sqrt{a'+\Omega_r}\dd a']\\
        &=\frac{2}{H_0}\qty[a\sqrt{a + \Omega_r} - \frac{2}{3}\qty(a+\Omega_r)^{\frac32} + \frac{2}{3}\Omega_r^{\frac{3}{2}}]\\
    a(0.1MeV) &= 2.725K/0.1MeV = 2.35\times 10^{-9}\\
    a(0.25eV) &= 2.725K/0.25eV = 9.39\times 10^{-4}&\text{书第四页(1.4)}\\
    H_0 &= 1.023h\times10^{-10}\rm{year}^{-1} = 7.37 \times10^{-11}\rm{year}^{-1}\\
\intertext{所以}
    t(a_{0.1MeV}) &= 4.187 \times 10 ^ {-6}\rm{yr.} \sim 132 \rm{sec.} \\ 
    t(a_{0.25eV}) &\sim 240000 \rm{yr.}
\end{align*}
% \section{Exercise 11.}
% \paragraph{(a)}
物质主导宇宙:
\begin{align*}
    \dv{a}{t} &= aH_0\sqrt{\frac{\rho}{\rho_{cr}}} = aH_0\sqrt{\frac{1}{a^3}}\\
    \dd t &= \frac{1}{H_0}\sqrt{a}\dd a\\
    \eta &= \int_0^t\frac{\dd t'}{a(t')} = \frac{1}{H_0}\int_0^a\frac{\sqrt{a'}}{a'}\dd a'\\
        &= \frac{2}{H_0}a^{\frac{1}{2}}\\
    \eta &\propto a^{\frac{1}{2}}\\
\intertext{辐射主导宇宙:}
    \dv{a}{t} &= aH_0\sqrt{\frac{\rho}{\rho_{cr}}} = aH_0\sqrt{\frac{1}{a^4}}\\
    \dd t &= \frac{1}{H_0}a\dd a\\
    \eta &= \int_0^t\frac{\dd t'}{a(t')} = \frac{1}{H_0}\int_0^a\dd a'\\
        &=\frac{a}{H_0}\\
    \eta &\propto a\\
\end{align*}
\paragraph{(b)}
\begin{align*}
    \dv{a}{t} &= aH_0\sqrt{\frac{\rho}{\rho_{cr}}} = aH_0\sqrt{\frac{\Omega_m}{a^3} + \frac{\Omega_r}{a^4}} = H_0\sqrt{\Omega_m}\sqrt{\frac{1}{a} + \frac{a_{eq}}{a^2}} & \text{书51页$\to\Omega_r = a_{eq}\Omega_m$}\\
    \dd t &= \frac{1}{\sqrt{H_0^2\Omega_m}}\frac{a\dd a}{\sqrt{a + a_{eq}}}\\
    \eta &= \frac{1}{\sqrt{H_0^2\Omega_m}}\int_0^a\frac{\dd a'}{\sqrt{a' + a_{eq}}}\\
        & = \frac{2}{\sqrt{H_0^2\Omega_m}}\qty(\sqrt{a + a_{eq}} - \sqrt{a_{eq}}) 
\end{align*}
% \section{Exercise 12.}\label{1:12}
% \begin{align*}
    \chi(a) &= \int_{a}^1 \frac{\dd a'}{a'^2H(a')} 
\intertext{在平坦, 物质主导的宇宙下:}
    H(a) &= H_0\sqrt{\frac{1}{a^3}}\\
    \chi(a) &= \int_{a}^1 \frac{\dd a}{H_0\sqrt{a}} = \frac{2}{H_0}\qty(1 - \sqrt{a})\\
    d_A(a) &= a\chi(a) = \frac{2a}{H_0}\qty(1-\sqrt{a})\\
    \theta(a) &= \frac{l}{d_A(a)} = \frac{lH_0}{2a\qty(1-\sqrt{a})} = \frac{1.668h\times10^{-6}}{2a\qty(1-\sqrt{a})}\\
    \theta(a_{z=0.1}) &= \theta(\frac{10}{11}) = 1.971h\times10^{-5}\sim4.07h''\\
    \theta(a_{z=1}) &= \theta(\frac{1}{2}) = 5.695h \times 10^{-6}\sim 1.17h''
\intertext{在平坦, $30\%$物质$70\%$暗能量的宇宙下:}
    H(a) &= H_0\sqrt{\frac{0.3}{a^3} + 0.7}\\
    d_A(a) &= a\chi(a) = \frac{a}{H_0}\int_a^1\frac{\dd a'}{\sqrt{0.3a' + 0.7a'^4}} \\
    d_A(\frac{10}{11}) &= 0.0888\frac{1}{H_0}\\
    d_A(0.5) &= 0.3857\frac{1}{H_0}\\
    \theta_{z=0.1} &= \frac{lH_0}{0.0888} = 1.8784h\times 10 ^{-5}\sim 3.87h''\\
    \theta_{z=0.1} &= \frac{lH_0}{0.3857} = 4.3246h\times 10 ^{-6}\sim 0.89h''\\
\end{align*}
% \section{Exercise 13.}
% \input{contents/chap2/ex_13.tex}
% \section{Exercise 14.}
% \paragraph{(a)}
已知$\rho$
\begin{align*}
    \rho_i &= g_i\int\frac{\dd^3 p}{\qty(2\pi)^3}f_i\qty(\va*{x},\va*{p})E(p)
\intertext{求$\mathcal{P}=?$}
    \mathcal{P}_i &= -g_i\int\frac{\dd^3 p}{\qty(2\pi)^3}f_i\qty(\va*{x},\va*{p})V\pdv{E(p)}{V}\\
    E(p) &= \sqrt{p^2+m^2} \quad p = 2\pi n V^{-\frac{1}{3}}\\
    \pdv{E(p)}{V} &= \frac{p}{E}\pdv{p}{V} = -\frac{p}{3E}pV^{-1}\\
    \mathcal{P}_i &= g_i\int\frac{\dd^3 p}{\qty(2\pi)^3}f_i\qty(\va*{x},\va*{p})\frac{p^2}{3E(p)}\\
\end{align*}
相对论粒子:$E(p)=p$
\begin{align*}
    \mathcal{P}_i &= g_i\int\frac{\dd^3 p}{\qty(2\pi)^3}f_i\qty(\va*{x},\va*{p})\frac{p^2}{3p} = \frac{1}{3}\rho_i\\
\end{align*}
\paragraph{(b)}
\begin{align*}
    \rho &= g\int\frac{\dd^3 p}{\qty(2\pi)^3}f\qty(\va*{x},\va*{p})E(p) = g\int\frac{p^2\dd p}{2\pi^2}f\qty(\va*{x},\va*{p})E(p)\\
    \mathcal{P} &= g\int\frac{\dd^3 p}{\qty(2\pi)^3}f\qty(\va*{x},\va*{p})\frac{p^2}{3E(p)} = g\int\frac{p^2\dd p}{2\pi^2}f\qty(\va*{x},\va*{p})\frac{p^2}{3E(p)}\\
    \dv{\mathcal{P}}{T} &= g\int\frac{p^2\dd p}{2\pi^2}\dv{f\qty(E/T)}{T}\frac{p^2}{3E(p)}\\
    \dv{f(E/T)}{T} &= -f'(E/T)\frac{E}{T^2} = -f'(E/T)\frac{E}{T}\dv{E/T}{E} = -\frac{E}{T}\dv{f(E/T)}{E} \\ 
    \dv{\mathcal{P}}{T} &= -g\int\frac{p^2\dd p}{2\pi^2}\dv{f\qty(E/T)}{E}\frac{p^2}{3T}\\
    &= -g\int\frac{p^2}{2\pi^2}\frac{E}{p}\frac{p^2}{3T}\dd f\qty(E/T) \quad (\dv{E}{p} = \frac{p}{E})\\
    &= g\int\frac{1}{2\pi^2}\frac{1}{3T}f(E/T)\dd Ep^3\\
    &= g\int\frac{1}{2\pi^2}\frac{1}{3T}f(E/T)(3p^2E\dd p + \frac{p^4}{E}\dd p)\\
    &=\frac{1}{T}\qty[g\int\frac{p^2\dd p}{2\pi^2}f(E/T)E + g\int \frac{p^2\dd p}{2\pi^2}f(E/T)\frac{p^2}{3E}]\\
    & = \frac{\rho + \mathcal{P}}{T}
\end{align*}
% \section{Exercise 15.}
% \begin{align*}
    T^0_0 &= g\int\frac{\dd P_1\dd P_2\dd P_3}{(2\pi)^3}(-\det[g_{\alpha\beta}]^{-1/2})P_0f(\va*{x}, \va*{p}, t)\\
    \rho &= g\int\frac{\dd P_1\dd P_2\dd P_3}{(2\pi)^3}a^{-3}P_0f(\va*{x}, \va*{p}, t)\\
    p^2 &= g^{ij}P_iP_j = \frac{1}{a^2}P_iP_i \to P_i = ap_i\\
    \rho &= g\int\frac{\dd p_1\dd p_2\dd p_3}{(2\pi)^3}Ef(\va*{x}, \va*{p}, t)\\
    & = g\int\frac{\dd^3 p}{(2\pi)^3}f(\va*{x}, \va*{p}, t)E\\
\end{align*}
% \section{Exercise 16.}
% \begin{align*}
    \chi(a) &= \int_a^1\frac{\dd a'}{a'^2H(a')}
    d_L(a) &= \frac{\chi(a)}{a}\\
    m-M &= 5\log\qty(\frac{d_L}{10\rm{pc}}) + K\\
\intertext{在物质主导的宇宙下:}
    \chi(a) &= \frac{2}{H_0}(1-\sqrt{a})  &&\text{(见\ref{1:12}推导)} \\
    d_L(a) &= \frac{\chi(a)}{a} = \frac{2}{H_0}\frac{1- \sqrt{a}}{a} = 2997.92h^{-1}  \frac{2 - 2\sqrt{a}}{a} \rm{Mpc.}\\
    d_L(z) &= 5995.84h^{-1}(1+z)\qty(1-\sqrt{\frac{1}{1+z}})\rm{Mpc.}\\
    m-M &= 5\log\qty[\qty(1+z)\qty(1 - \sqrt{\frac{1}{1+z}})] + 5\log(5.99584h^{-1}\times10^8) + K\\
\intertext{在平坦, $30\%$物质$70\%$暗能量的宇宙下:}
    \chi(a) &= \frac{1}{H_0}\int_a^1\frac{\dd a'}{\sqrt{0.3a' + 0.7a'^4}} \\
    \chi(z) &= \frac{1}{H_0}\int_{\frac{1}{1+z}}^1\frac{\dd a'}{\sqrt{0.3a' + 0.7a'^4}} \\
    d_L(z) &=  2.99792h^{-1}\times10^9(1+z)\int_{\frac{1}{1+z}}^1\frac{\dd a'}{\sqrt{0.3a' + 0.7a'^4}}\rm{pc.} \\
    m - M &= 5\log\qty[(1+z)\int_{\frac{1}{1+z}}^1\frac{\dd a'}{\sqrt{0.3a' + 0.7a'^4}}] + 5\log\qty(2.99792h^{-1}\times10^8) + K
\end{align*}
\begin{tikzpicture}[scale=1.5]
    \begin{semilogxaxis}[
        title=a,
        xlabel={$z$},
        ylabel={$m-M$},
        legend entries={no $\Lambda$, with $\Lambda$},
        legend pos=south east,
        log ticks with fixed point,
    ]
        \addplot[no markers] table[x=z, y=m1, col sep=comma]{./contents/chap2/data16_2.csv};
        \addplot[no markers, dashed] table[x=z, y=m2, col sep=comma]{./contents/chap2/data16_2.csv};
    \end{semilogxaxis}
\end{tikzpicture}
% \section{Exercise 17.}
% \begin{align*}
    s &= \frac{4\rho}{3T} = \frac{4g}{3T}\int\frac{\dd ^3 p}{(2\pi)^3} \frac{E}{e^{E/T} \pm 1}\\
    &=\frac{4g}{3T}\int_0^\infty\frac{p^2\dd p}{2\pi^2}\frac{p}{e^{p/T} \pm 1}\\
    &=\frac{4gT^3}{3}\frac{1}{2\pi^2}\int_0^\infty\dd x \frac{x^3}{e^x \pm 1}\\
\intertext{费米子:}
    &=\frac{2gT^3}{3\pi^2}(1 - 2^{-3})\Gamma(4)\zeta(4) = \frac{7}{8}\frac{2\pi^2gT^3}{45}\\
\intertext{玻色子:}
    &=\frac{2gT^3}{3\pi^2}\Gamma(4)\zeta(4) = \frac{2\pi^2gT^3}{45}\\
    s &= \frac{2\pi^2}{45}\qty[\sum_{i=\rm{bosons}}g_iT_i^3 + \frac{7}{8}\sum_{i=\rm{fermions}}g_iT_i^3]
\intertext{如果$m\neq0$}
    \rho &= \frac{4g}{3T}\int_0^\infty\frac{p^2\dd p}{2\pi^2}\frac{\sqrt{p^2+m^2}}{e^{\sqrt{p^2+m^2}/T} \pm 1} \\
    &\sim e^{-m}\frac{4g}{3T}\int_0^\infty\frac{p^2\dd p}{2\pi^2}\frac{\sqrt{p^2+m^2}}{e^{p/T} \pm 1}\\
\end{align*}
% \section{Exercise 18.}
% \begin{align*}
\intertext{光子:}
    n_\gamma &= g_\gamma\int \frac{\dd^3 p}{(2\pi)^3}\frac{1}{e^{p/T_\gamma} - 1}\\
        &= g_\gamma \int_0^\infty\frac{p^2\dd p}{2\pi^2}\frac{1}{e^{p/T_\gamma} - 1}\\
        &=\frac{g_\gamma T_\gamma^3}{2\pi^2}\Gamma(3)\zeta(3) = 0.1218gT_\gamma^3 \\
    n_\gamma &= 0.1218g_\gamma T_\gamma^3 = 410.5cm^{-3} \quad T_\gamma = 2.725K\\
\intertext{中微子:}
    n_\nu &= g_\nu \int_0^\infty\frac{p^2\dd p}{2\pi^2}\frac{1}{e^{p/T_\nu} + 1}\\
        &=\frac{3}{4}\frac{g_\nu T_\nu^3}{2\pi^2}\Gamma(3)\zeta(3) \\
        &=\frac{3}{4}\frac{g_\nu}{2\pi^2}\frac{4}{11}T_\gamma^3 = \frac{3}{11}\frac{g_\gamma}{2\pi^2}T_\gamma^3 = \frac{3}{11}n_\gamma \quad g_\gamma = g_\nu = 2;T_\nu = \qty(\frac{4}{11})^{1/3}T_\gamma\\
        &=112cm^{-3}\\
\end{align*}
\section{Exercise 19.}
\begin{align*}
    \rho_\nu &= 2 * \frac{7}{8}\qty(\frac{4}{11})^{4/3}\rho_\gamma\\
    \Omega_\nu &= \frac{2}{3} \frac{1.68\times10^{-5}}{h^2}\\
    \Omega_r &= \frac{3.59\times10^{-5}}{h^2}\\
    \frac{\Omega_r}{a_{eq}^4} &= \frac{\Omega_m}{a_{eq}^3}\\
    a_{eq} &= \frac{\Omega_r}{\Omega_m} = \frac{3.59\times10^{-5}}{\Omega_m h^2}
\end{align*}

\chapter{Beyond Equilibrium}
\end{document}