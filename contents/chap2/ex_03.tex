$g_{\mu\nu}$的形式为:
$$g_{\mu\nu} = \mqty(\dmat[0]{-1 - 2 \phi,1 - 2\phi,1- 2\phi,1- 2\phi})$$
\paragraph{(a)}
利用公式\ref{chris}得
\begin{align*}
    \Gamma^0_{00} &= \frac{g^{0\nu}}{2}\qty[\pdv{g_{0\nu}}{x^0}+\pdv{g_{0\nu}}{x^0}-\pdv{g_{00}}{x^\nu}]\\
        &=\frac{g^{00}}{2}\qty[\pdv{g_{00}}{x^0}] = \frac{1}{1 + 2\phi} \pdv{\phi}{t}\\
        &\sim\pdv{\phi}{t}
\end{align*}
\begin{align*}
    \Gamma^i_{00} &= \frac{g^{i\nu}}{2}\qty[\pdv{g_{0\nu}}{x^0}+\pdv{g_{0\nu}}{x^0}-\pdv{g_{00}}{x^\nu}]\\
    & = \frac{\delta^{ij} g^{11}}{2}\qty[\pdv{g_{0j}}{x^0}+\pdv{g_{0j}}{x^0}-\pdv{g_{00}}{x^j}] \\
    & = -\frac{\delta^{ij} g^{11}}{2}\pdv{g_{00}}{x^j} = \frac{1}{1-2\phi} \delta^{ij}\pdv{\phi}{x^j}\\
    & \sim \delta^{ij} \pdv{\phi}{x^j}
\end{align*}


\paragraph{(b)}
利用测地线方程\ref{4dgeodesic}, 在时间分量上有:
\begin{align*}
    \dv[2]{x^0}{\lambda} + \Gamma^0_{\alpha\beta}\dv{x^\alpha}{\lambda}\dv{x^\beta}{\lambda} = 0
\end{align*}
计算$\Gamma$